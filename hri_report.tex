\documentclass[conference]{IEEEtran}

% Include packages etc.
\usepackage[square,sort,comma,numbers]{natbib}
\bibliographystyle{unsrtnat}
\usepackage{amsmath,amssymb,amsfonts}
\usepackage{algorithmic}
\usepackage{graphicx}
\usepackage{textcomp}
\usepackage{xcolor}
\usepackage{fancyhdr}
\usepackage{array}
\usepackage{booktabs}
 
\pagestyle{fancy}
\fancyhf{}
\rhead{Analysis of Cognitive Load in Human Robot Interactions}
\lhead{University of the West of England }
\rfoot{Page \thepage}

\def\BibTeX{{\rm B\kern-.05em{\sc i\kern-.025em b}\kern-.08em
    T\kern-.1667em\lower.7ex\hbox{E}\kern-.125emX}}
\begin{document}

% Title and Subtitles
\title{Analysis of Cognitive Load in Human Robot Interactions}

% Author
\author{
	\IEEEauthorblockA{Simon Bluett, Alfie Sargent, Natalia Gonz\'alez Cadiente, Xuan Wang, Jinze Ding \\
	\textit{Human Robot Interaction - University of the West of England}}
}

\maketitle


% _________________________________________________________________________________
%% Abstract
% ^^^^^^^^^^^^^^^^^^^^^^^^^^^^^^^^^^^^^^^^^^^^^^^^^^^^^^^^^^^^^^^^^^^^^^^^^^^^^^^^^
\begin{abstract}
Include background, methods, results and a conclusion.
\end{abstract}

\begin{IEEEkeywords}
cognitive load, ambiguity, human-robot interaction
\end{IEEEkeywords}


% _________________________________________________________________________________
%% Introduction (PERSONAL)
% ^^^^^^^^^^^^^^^^^^^^^^^^^^^^^^^^^^^^^^^^^^^^^^^^^^^^^^^^^^^^^^^^^^^^^^^^^^^^^^^^^
\section{Introduction}
Robotics is quickly becoming a part of every day life for many people with devices such as the Roomba \cite{web:roomba} \& Alfawise Magnetic \cite{web:alfawise} becoming increasingly popular. They offer the ability to autonomously complete mundane tasks like cleaning the house so that the home owner doesn’t have too; essentially increasing the standard of living for an individual. They also have other benefits such as helping disabled individuals with assistive robotics\cite{mataric2007socially} and educating children with autism \cite{shamsuddin2012initial}. All of these examples require the human to interact in the robot in some way and then for the robot to respond; human robot interaction. \newline
As robotics begins to become more quotidian, the way in which people interact with these robots should be investigated to ensure the best possible design is implemented. Cognitive load refers to the used amount of working memory resources when completing a task or interacting in some way. For more robotic devices to be welcomed into the general populace and consequently potentially improve the standard of living of the user, the relation of cognitive load during human robot interactions should be researched. \newline

In this study we aim to investigate the cognitive load of a user during human robot interaction with a Nao robot \cite{web:NaoRobot} in comparison to a video-based agent. This will be done by having a variety of participants complete a task with instructions given by a physical Nao robot and a computer, monitoring physiological data during the task as well as a questionnaire at the end to determine cognitive load. 

% _________________________________________________________________________________
%% Related Work (PERSONAL)
% ^^^^^^^^^^^^^^^^^^^^^^^^^^^^^^^^^^^^^^^^^^^^^^^^^^^^^^^^^^^^^^^^^^^^^^^^^^^^^^^^^
\section{Related Work}
The following section details pre-existing work in the two main areas that are being investigated by this user study: Human Robot Interaction in comparison to Video Displayed Agents, and Cognitive Load Theory and its physiological effects in Human Robot Interaction.

\subsection{Cognitive Load Theory}
Based on prevailing cognitive load theory \cite{deleeuw2008comparison,de2010cognitive,sweller1994cognitive}, there are 3 types of cognitive processing when learning: extraneous processing, intrinsic processing and germane processing (see Table~\ref{Tabel:1}).
\begin{table*}[h!]
    \caption{A table detailing the 3 types of cognitive load.}
    \label{Tabel:1}
    \centering
    \begin{tabular}{| m{8em} | m{8cm}| m{6cm} |}
     \hline
     \textbf{Cognition type} & Description & Example \\ [0.5ex] 
      \hline\hline
       \textbf{Extraneous} & Predominately caused by inappropriate or frustrating instructional design that ignores the natural working limits of human memory. & An educational book that isn't well laid out, requiring the read to go back and forth. \\
         \hline
        \textbf{Intrinsic} & Caused by the natural structure and complexity of the topic being covered; the inherent load required to learn something. & The calculation of  $2+2=4$ has a lower load than more advanced mathematics like calculus.\\
         \hline
       \textbf{Germane} & A deeper cognitive learning process that involves the individual mentally organising what has been learned for cognitive access later; this is often described as being organised into a schema \cite{paas2003cognitive}.& Understanding the relationship between topics like how the algorithms used in one programming language can be used in another; forming a connection in long term memory and strengthen that connection.\\
         \hline
    \end{tabular}
\end{table*}
\newline
 Each type of cognitive load will be measured in this user study. However, researchers predominately observe cognitive load using questionnaires and other non-statistical evaluations primarily because it is difficult to measure cognitive load through biometric data without influencing the results. 

\subsection{Cognitive Load Measurements}
Cognitive Load is difficult to physiologically measure since it is affected by a variety of conditions: the task being done, how its being done, the experience of the user etc. To generate reliable results for this study, the environment must be precisely controlled and all external and internal validation influences minimised.

\citet{deleeuw2008comparison} investigated the cognitive load of 155 college students with limited engineering knowledge when attending a lesson on electric motors. The lesson involved a secondary task where the participants responded to a distraction task throughout the experiment as well as self-report their cognitive load throughout. The cognitive load was measured by: a preliminary questionnaire, response time to the secondary task, 8 self-report scales throughout experimentation and a final questionnaire. The results created are consistent with the prevailing theory at the time \cite{ayres2006using}; that different features of cognitive load can be accessed by different measurements of cognitive load. However, there was not much variance in the type of cognitive load and this could be due to one method of measurement; the self-reporting of cognitive load. By reporting cognitive load during the experiment the results are no longer reliable because the participants cognition is being manipulated during the task by making them do this report. The measuring techniques in this study will emulate some parts of the researched experiment but it will not require participants to report their own cognition this many times during the study to avoid tampering with other measurements. Furthermore, the experiment discussed does not use any biometric measures of cognition. To improve/develop upon the findings of the paper, biometric data will be a considered measurement.

Many papers have examined the physiological effects that cognitive load can have on individuals such as eye position tracking \cite{palinko2010estimating}, relative blood flow \cite{ikehara2005assessing} and pupil size \cite{partala2003pupil}.

The most prominent measurement which suggests a proportionality with cognitive load is the galvanic skin response (GSR) of an individual---a measure of skin conductance \cite{shi2007galvanic,nourbakhsh2012using}. \citet{ikehara2005assessing} performed a study using GSR, consisting of 13 male US air force volunteers aged between 18-21 with each participant being studied individually for 1 hour. The experiment had participants select fractions that appeared on the left of a computer screen and moved to the right. The participant was tasked with selecting fractions greater than a 1/3 before it hit the right side of the screen (Figure~\ref{fig:MovingfracitonsTask}).

\begin{figure}[h!]
    \centering
    \includegraphics[width=0.95\linewidth]{figures/ScreenCaptureFractions.PNG}
    \caption{Screen capture of the moving fractions task \cite{ikehara2005assessing}}
    \label{fig:MovingfracitonsTask}
\end{figure}

The participant would gain points for choosing correctly, lose points for not selecting in time and lose points for selecting the wrong fraction. The task had 2 "difficulties", one with 16 simple fractions and one with 56 complex fractions.

The GSR results showed that the device was capable of detecting a decrease in sweat cell arousal as the complexity of the task increased. The study used a bespoke physiological measuring device with little detail on its development. The results from this device have not been proven reliable; the study described throughout this paper will use a professional sensor to ensure reliability. The sensor itself was applied to the participants left hand, with no indication of which hand was preferred by the user, potentially changing the results especially considering the experiment required participants to actively use their hand. In  the study described in this paper, the sensor will be placed on the non-dominant hand, as well as ensuring the participants hand is kept stationary.

\subsection{HRI vs Video Displayed Agents}
Investigating how humans interact with robots in comparison to a computer is one of the main objectives of this study, and thus current research must be investigated prior to the study being run. \newline

One study found how a robot's physical presence can affect a humans judgement when handling physical objects \cite{bainbridge2011benefits}. In the experiment, 65 university staff, graduates and undergraduates were tasked with relocating books in a room; 22 had the robot physically with them, 22 had a live feed of the robot and 21 had an augmented video of the robot. The results show the participants had a greater "respect" for the robots sense of space and would actively go out of their way to avoid the robots space. \newline

Furthermore, the study showed an increase in "trust" if the task was given by a physical robot. Although this experiment was focused on social interaction and not cognitive load of participants, the results gathered are still relevant to the user study being created. The mention of personal space and trust being greater with the presence of a robot may be indicative of cognitive load as well; this will be studied further in the experiment being designed.
\begin{figure}[h]
    \centering
    \includegraphics[width=0.95\linewidth]{figures/Nico.PNG}
    \caption{\label{fig:my_label}Nico, the robot used by \citet{bainbridge2011benefits}. Participants were more willing to follow orders given by Nico}
\end{figure} 
\newline

% _________________________________________________________________________________
%% Methods (GROUP)
% ^^^^^^^^^^^^^^^^^^^^^^^^^^^^^^^^^^^^^^^^^^^^^^^^^^^^^^^^^^^^^^^^^^^^^^^^^^^^^^^^^
\section{Methods}

% - - - - - - - - - - - - - - - - - - - - -
\subsection{Hypotheses}
The main research question of the study is: Does the introduction of a physically embodied robot increase cognitive load when interacting with humans to complete a complex task? Our hypothesis is that introducing a robot into the learning process of a complex task will increase cognitive load and therefore hinder the ability to complete the task.

The way in which we measure the hypothesis is to firstly ask participants to fill out questionnaires indicating their stresses, preferences and overall thoughts on their experience. Secondly, we conduct physiological tests looking at skin conductance, heart rate and reaction times to an awareness test; slow reaction times and increased heart rate will indicate increased cognitive load of the participant. 

% - - - - - - - - - - - - - - - - - - - - -
\subsection{Validity and  Reliability}

\subsubsection{Internal validity} 
Internal validity is controlled by randomising the order of the tasks and ensuring participants have not seen the study prior to taking part. This means participants aren't affected due to learning how the robot behaves and preparing for the study. Selection effects are avoided by randomly selecting participants so that not all participants have the same knowledge and background. For this study, it was found that many of the users had previous experience with some form of robot in their life and therefore this may affect the internal validity. To remove experimental bias in which those conducting the study influence the outcome, the same guidelines are provided for both robot and computer scenarios, and the computer uses the same voice as the robot when giving instructions. The guidelines are given on paper to avoid indicating preference to the participant.

\subsubsection{External validity}
Environmental influences include other studies being run in the same room, many people watching, and outside noise that may cause disturbance to the participant. In real-life scenarios, outside noise and other humans nearby may increase the generalisability of this study. External validity could be further increased by testing different sub-groups, for example including participants of a wider age range and background.

\subsubsection{Reliability}
9 participants were studied in this research. Participants had a variety of backgrounds and experiences with both the oscilloscope equipment and robot interaction. Studying more participants would increase the reliability and repeatability of the study, providing more conclusive results. Uncontrolled variations also affected the reliability of the study. Most users had some experience with robots or the oscilloscope equipment. More reliable results would include users with varying prior experience.

\subsection{User Study Design}
The study was designed to take into consideration the three categories of cognitive load. The task consisted of operating a oscilloscope and waveform generator, whose complex array of buttons and dials should provide an intrinsic cognitive load to people who are not familiar with the equipment. 

The extraneous cognitive load comes from the demands by the teacher or instructions; how the information is presented. This type of cognitive load inhibits the humans ability to learn and successfully complete tasks and is therefore the type of load a robot may influence. We deliver some instructions via a robot, and others via a computer to analyse the changes in cognitive load in human-robot interaction.

A within user study design is chosen due to the limited number of users available to participate within the time frame. This is to allow us to gather sufficient results for analysis. The study works by alternating which test condition the user is exposed to first; robot or computer.

% - - - - - - - - - - - - - - - - - - - - -
\subsection{User Study Procedure}
The participant is guided through the task of creating a waveform using the equipment provided in 8 steps. Four steps are presented by the robot, and four by the computer. Throughout this, the participant must complete an awareness task.

\subsubsection{Awareness Task}
We set a cognitive load baseline by using an awareness task. The users are asked to pay attention to a computer whose screen changes colour periodically. When the screen changes to black, the participant must press the spacebar. The users heart rate and skin conductance are measured, and the number of correct and incorrect key presses by the user are recorded.
This task is used before the test to create a baseline, and during both computer and robot stages of the study.

\subsubsection{Questionnaires}
Three questionnaires are used throughout the study:
\begin{itemize}
    \item Pre-study Questionnaire: Questions regarding age, demographic, equipment experience, and robot experience
    \item Mid- and Post-study Questionnaire: The same questionnaire presented separately, after the robot condition and after the computer condition. Questions regarding frustration levels, understanding of the task, how well they have learned the task. (NASA TLX style)
    \item Mid- and Post-study Interview: Pre-written questions regarding opinions of the robot/computer, suggested improvements, and thoughts on the task itself.
\end{itemize}

% - - - - - - - - - - - - - - - - - - - - -
\subsection{Dependent Measures}
\begin{itemize}
    \item NASA TLX style questionnaire \cite{hart1988development}
    \item Galvanic Skin Response (GSR) and Heart Rate
    \item Awareness Test
    \item Time taken to complete the tasks
    \item Number of times the instructions had to be repeated
    \item The successful completion of each of the tasks
\end{itemize}
\subsection{Independent Variables}
\begin{itemize}
    \item Robot or Computer first
\end{itemize}

% _________________________________________________________________________________
% Datasets and Results (GROUP)
% ^^^^^^^^^^^^^^^^^^^^^^^^^^^^^^^^^^^^^^^^^^^^^^^^^^^^^^^^^^^^^^^^^^^^^^^^^^^^^^^^^
\section{Datasets and Results}

% - - - - - - - - - - - - - - - - - - - - -
\subsection{Participants}

The experiment was conducted at the Bristol Robotics Laboratory and it was attended by randomly selected 5 men and 4 women. All participants were aged between 21-25, and 66.7\% (N=5) of them were native English speakers. The pre-survey questionnaire showed that three of them had no relevant human-robot interaction experience, and had not heard of the robot `NAO'. Only one participant indicated that they would not be interested in having a service robot in their home. Since the participants’ academic background and training could also influence their cognitive load during the study, the participants were asked about their familiarity with the devices (oscilloscope and waveform generator) being used throughout the study; more than half were unfamiliar with the the devices being used. The duration of each trial was approximately 15 minutes. During the course of the experiment, the basic physiological characteristics of the participants, such as heart rate, skin conductance, and skin resistance were recorded using a \textit{Shimmer} sensor. In the subsequent sections, this physiological data and the results from the questionnaires are recorded, and the overall trends analysed. 

% - - - - - - - - - - - - - - - - - - - - -
\subsection{Physiological Data}
Skin conductive, skin resistance and heart rate were recorded for 9 participants during both robot and computer experiments. Baseline data was recorded for 30 seconds before each of the experiments. The change of those parameters for both robot and computer instruction relative to baseline are presented. \newline

\subsubsection{Galvanic skin conductance} The result showed that the skin conductance difference of the robot instructions ($\mu$ = 2.23, $\sigma$ = 3.09) were higher than that of the computer instructions ($\mu$ = 1.59, $\sigma$ = 1.43). The average skin conductance of 6 of the participants during the robot instruction was higher than the computer instruction. Figure \ref{fig:skin_conductance} illustrates the skin conductance of each of the participants.

\begin{figure}[h]
	\flushleft 
	\includegraphics[width=1.05\linewidth]{figures/conductance.png}
	\caption{\label{fig:skin_conductance}Galvanic skin conductance values for all participants.}  
\end{figure}


\subsubsection{Heart rate} The average heart rate of participants during robot instructions ($\mu$ = 88.64BPM, $\sigma$ = 1.596) was on par with those measured during the computer instructions ($\mu$ = 88.62BPM, $\sigma$ = 2.495). The average heart rate of participants is shown in Figure \ref{fig:heart_rate}.

\begin{figure}[h]
	\flushleft 
	\includegraphics[width=1.05\linewidth]{figures/heartrate.png}
	\caption{\label{fig:heart_rate}Heart rate values for all participants.}  
\end{figure}

% - - - - - - - - - - - - - - - - - - - - -
\subsection{Awareness Task}
Figure \ref{fig:awareness_reaction} illustrates the average reaction times of each participant to pressing the keyboard when the awareness test screen turned black. The results for the first four subjects was not recorded properly, so they have been excluded. The reaction time tended to be slower during the robot scenario, relative to the computer (Avg. increase of +488.5ms, $\sigma$ = 837.7). The subjects detected a lower percentage of the black screens during the robot instructions ($\mu$ = 0.38, $\sigma$  = 0.387) compared to the computer ($\mu$ = 0.5, $\sigma$  = 0.400).

\begin{figure}[h]
	\flushleft 
	\includegraphics[width=1\linewidth]{figures/awareness_reactions.png}
	\caption{\label{fig:awareness_reaction}Reaction time of participants responding to the awareness task.}  
\end{figure}

% - - - - - - - - - - - - - - - - - - - - -
\subsection{Instruction Completion Times}

Figure \ref{fig:instruction_time_diff} shows the average time differences of completing the computer instructions, relative to the robot instructions. In general, the computer instructions were completed more quickly ($\mu$ = -2912ms, $\sigma$ = 6328). It was also found that the second set of instructions tended to take more time to complete, relative to the first set ($\mu$ = +5202ms, $\sigma$ = 4125).

\begin{figure}[h]
	\flushleft 
	\includegraphics[width=1\linewidth]{figures/difference.png}
	\caption{\label{fig:instruction_time_diff}Average time difference of each participants to complete the computer instructions relative to the robot instructions.}  
\end{figure}

% - - - - - - - - - - - - - - - - - - - - -
\subsection{Post experiment questionnaire}
For each robot and computer experiment, the participants answered 7 question by the degree from 1 to 5. The average point value of the questions are presented Table \ref{tab:questionairre}.   

\begin{table}[h]
\caption{\label{tab:questionairre}Average scores}
\begin{tabular}{|c|c|c|c|}
\hline
\multicolumn{1}{l|}{No} & \multicolumn{1}{l|}{}                                                                                                          & \multicolumn{1}{l|}{Robot} & \multicolumn{1}{l|}{Computer} \\
\hline \hline
1                       & \begin{tabular}[c]{@{}c@{}}How would you rate your experience with \\ the equipment used? \end{tabular} & 3.875                      & 4.375                         \\ \hline
2                       & \begin{tabular}[c]{@{}c@{}}How difficult did you find the tasks?\end{tabular}                                               & 2.625                      & 1.75                          \\ \hline
3                       & Were the instructions given clear?                                                                                             & 3.5                        & 3.875                         \\ \hline
4                       & \begin{tabular}[c]{@{}c@{}}How frustrated did you feel during \\ the tasks?\end{tabular}                                       & 2.375                      & 1.75                          \\ \hline
5                       & \begin{tabular}[c]{@{}c@{}}How competently do you think you\\  completed the tasks assigned?\end{tabular}                      & 3.5                        & 4.125                         \\ \hline
6                       & \begin{tabular}[c]{@{}c@{}}How much more guidance would you need \\ if you were to do the task again?\end{tabular}             & 2.75                       & 1.875                         \\ \hline
7                       & \begin{tabular}[c]{@{}c@{}}How confident are you now in\\  completing the task unassisted?\end{tabular}                       & 3.5                        & 4                             \\ \hline

\end{tabular}
\end{table}

The distribution of scores for each participant is presented in Figure \ref{fig:questionnaire_distrib_1} and Figure \ref{fig:questionnaire_distrib_2}; one person's results is not showed because he/she didn't complete the task successfully and most questions were rated as unknown.

\begin{figure}[h]
	\flushleft 
	\includegraphics[width=1.05\linewidth]{figures/first.png}
	\caption{\label{fig:questionnaire_distrib_1}Questionnaire result for first group of participants.}
\end{figure}

\begin{figure}[h]
	\flushleft 
	\includegraphics[width=1.05\linewidth]{figures/last.png}
	\caption{\label{fig:questionnaire_distrib_2}Questionnaire result for second group of participants.}
\end{figure}

% - - - - - - - - - - - - - - - - - - - - -
\subsection{Post Experiment Interview}
Five participants claimed no significant difference between the instructions provided by the robot and the computer, but due to the complexity of the mission, they indicated that they did not pay much attention to the robot. In contrast, the questionnaire showed that the computer instructions required less attention. Only one person preferred the robot interaction to the computer one. Two participants stated that their understanding of the instructions was aided by the NAO’s gestures, while two other participants stated that the robot's instructions were hard to follow. 


% _________________________________________________________________________________
%% Discussion (PERSONAL)
% ^^^^^^^^^^^^^^^^^^^^^^^^^^^^^^^^^^^^^^^^^^^^^^^^^^^^^^^^^^^^^^^^^^^^^^^^^^^^^^^^^
\section{Discussion}
Five participants claimed no significant difference between the instructions provided by the robot and the computer, but due to the complexity of the mission, they indicated that they did not pay much attention to the robot. In contrast, the questionnaire showed that the computer instructions required less attention. Only one person preferred the robot interaction to the computer one. Two participants stated that their understanding of the instructions was aided by the NAO’s gestures, while two other participants stated that the robot's instructions were hard to follow. 
% _________________________________________________________________________________
%% Conclusion (PERSONAL)
% ^^^^^^^^^^^^^^^^^^^^^^^^^^^^^^^^^^^^^^^^^^^^^^^^^^^^^^^^^^^^^^^^^^^^^^^^^^^^^^^^^
\section{Conclusion}




% _________________________________________________________________________________
\bibliography{refs}
\end{document}

